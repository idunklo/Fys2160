  \documentclass[a4paper,norsk,12pt,oneside]{article}
% To use norwegian
\usepackage[utf8]{inputenc}
\usepackage[T1]{fontenc}
\usepackage[norsk]{babel}
% math
\usepackage{amsmath}
\usepackage{amsfonts}
\usepackage{amssymb}

\usepackage{graphicx} % images
\usepackage{float}
\usepackage{enumerate}
\usepackage{fancyvrb} % code
\usepackage{algorithm2e} % algorithm
% For listing
\usepackage{listings} % code with color
\usepackage{courier}
\usepackage{caption}
\usepackage{color}

\title{Fys2160, Oblig 1}
\author{Idun Kløvstad}
\date{\today}  

% Command for L'Hopital's rule (requires extarrows-package)
\newcommand{\Heq}[1]{\xlongequal[\mathrm{L'H}]{\left[#1\right]}}

% Double underline
\newcommand{\dul}[1]{\underline{\underline{#1}}}

% Custom operators
\DeclareMathOperator{\nul}{Nul\,}
\DeclareMathOperator{\Proj}{Proj\,}
\DeclareMathOperator{\Sp}{Sp\,}
\DeclareMathOperator{\res}{Res\,}
\DeclareMathOperator{\Log}{Log\,}

% Allow displayed page breaks
\allowdisplaybreaks

% Settings for listings
\definecolor{dkgreen}{rgb}{0,0.6,0}
\definecolor{gray}{rgb}{0.5,0.5,0.5}
\definecolor{mauve}{rgb}{0.58,0,0.82}

\lstset{%
  %language=python,                     % the language of the code
  basicstyle=\footnotesize\ttfamily,   % the size of the fonts that are used for the code
  numbers=left,                        % where to put the line-numbers
  numberstyle=\tiny,                   % the style that is used for the line-numbers
  stepnumber=2,                        % the step between two line-numbers. If it's 1, each line will be numbered
  numbersep=5pt,                       % how far the line-numbers are from the code
  extendedchars=true,
  %backgroundcolor=\color{white},       % choose the background color. You must add \usepackage{color}
  %showspaces=false,                    % show spaces adding particular underscores
  showstringspaces=false,              % underline spaces within strings
  %showtabs=false,                      % show tabs within strings adding particular underscores
  %frame=single,                        % adds a frame around the code
  frame=b,                             % adds a line at the bottom
  %rulecolor=\color{black},             % if not set, the frame-color may be changed on line-breaks within not-black text (e.g. commens (green here))
  tabsize=2,                           % sets default tabsize to 2 spaces
  %captionpos=b,                        % sets the caption-position to bottom
  breaklines=true,                     % sets automatic line breaking
  breakatwhitespace=false,             % sets if automatic breaks should only happen at whitespace
  %title=\lstname,                      % show the filename of files included with \lstinputlisting; also try caption instead of title
  keywordstyle=\color{blue},           % keyword style
  commentstyle=\color{dkgreen}\textit, % comment style
  stringstyle=\color{mauve},           % string literal style
  %escapeinside={\%*}{*)},              % if you want to add LaTeX within your code
  %morekeywords={*,...},                % if you want to add more keywords to the set
  xleftmargin=-20pt,
  xrightmargin=-20pt,
  framexleftmargin=19pt,
  framexbottommargin=4pt,
  framexrightmargin=21pt,
} 

% caption for listing
\newlength{\mycapwidth}\setlength{\mycapwidth}{\textwidth}
\addtolength{\mycapwidth}{75pt}
\DeclareCaptionFont{white}{\color{white}}
\DeclareCaptionFormat{listing}{\colorbox[cmyk]{0, 0, 0, 0.8}
    {\parbox{\mycapwidth}{\hspace{15pt}#1#2#3}}}
\captionsetup[lstlisting]{format=listing,labelfont=white,textfont=white, 
    singlelinecheck=false, margin=-40pt, font={bf,footnotesize}}


\begin{document}
\maketitle
\newpage 

\section{Part 1: The Einstein crystal}
\begin{enumerate}[a)]
    \item 

        We are looking at a system with \(N = 3\) oscillators and \(q = 3\). 
        For this system the possible microstates will be 

        \begin{tabular}{c | l | c | r }
            \label{pos_mikro}
            \(\#\) & \(n_1\) & \(n_2\) &\(n_3\) \\
            1 &1 & 1 & 1 \\
            2 & 1 & 2 & 0 \\
            3 & 1 & 0 & 2 \\
            4 & 2 & 0 & 1 \\
            5 & 2 & 1 & 0 \\
            6 & 0 & 1 & 2 \\
            7 & 0 & 2 & 1 \\
            8 & 0 & 0 & 3 \\
            9 & 0 & 3 & 0 \\
            10 & 3 & 0 & 0 \\
            \end{tabular}

    \item

        By using the general formula for the number of microstates, equation \ref{eq:1}, we 
        see that the number of microstates shown in table ~\ref{pos_mikro} are consistent 
        with the  answer from the equation. 

        \begin{align}
            \label{eq:1}
            \Omega(N,q) &= \binom{q + N -1}{q} = \frac{(q + N - 1)!}{q!(N - 1)!}\\
            \frac{5!}{3!2!} &= 10
        \end{align}

    \item

        For the following task we will look at two isolated Einstein crystals. 

        If subsystem A has \(N_A = 2\) and \(q_A = 5\) and subsystem B has \(N_B = 2\) and 
        \(q_B = 1\), the possible microstates is shown in the table below. 

        \begin{tabular}{c | l | c | r | c }
            \label{pos_mikro}
            \(\#\) & \(n_{A1}\) & \(n_{A2}\) &\(n_{B1}\) & \(n_{B2}\) \\
            1 & 0 & 5 & 0 & 1 \\
            2 & 1 & 4 & 0 & 1 \\
            3 & 2 & 3 & 0 & 1 \\
            4 & 3 & 2 & 0 & 1 \\
            5 & 4 & 1 & 0 & 1 \\
            6 & 5 & 0 & 0 & 1 \\
            7 & 0 & 5 & 1 & 0 \\
            8 & 1 & 4 & 1 & 0 \\
            9 & 2 & 3 & 1 & 0 \\
            10 & 3 & 2 & 1 & 0 \\
            11 & 4 & 1 & 1 & 0 \\
            12 & 5 & 0 & 1 & 0 \\
            \end{tabular}

    \item

        The two systems, A and B, are put in thermal contact. The total energy
        \(q = q_A + q_B = 6\) is constant, but can be distributed between the two
        systems.

        For \(N_A = 2\),  \(N_B = 2\) and \(q = 6\) there are seven different 
        macrostates of the system. The different macrostates are given by the
        possible values for \(q_A\) and \(q_B\) and is listed in the table below.

        \begin{tabular}{c | l | r}
            \label{pos_mikro}
            \(\#\) & \(n_1\) & \(n_2\) \\
            1 & 0 & 6 \\
            2 & 1 & 5 \\
            3 & 2 & 4 \\
            4 & 3 & 3 \\
            5 & 4 & 2 \\
            6 & 5 & 1 \\
            7 & 6 & 0 \\
            \end{tabular}

    \item

        For each possible macrostate \(q_a\) there are a number of compatible microstates. 
        The number of possible microstates for each macrostate is found by using the
        following program. 

        \lstinputlisting[language=Python, title = Calculating number of possible microstates for each macrostate]{Einstein_crystal.py}    

        Assuming that all microstates are equally probable, running the program also
        gives the probabilities of each macrostate.

    \item

        Before thermal contact, the system had 12 microstates, while after thermal 
        contact, it had 84. From this we can assume that a system has a lot more
        microstates after thermal contact. 

    \item

        \begin{figure}[H]
            \centering % center the image horizontally
            \includegraphics[width=0.9\textwidth]{part1.png}
            \caption{The probability, \(P_A\) as a function of \(q_A\) for a system with \(N_A = 50\), \(N_B = 50\) and \(q = 100\)} 
            \end{figure}  

        From the calculations in the program, we find that:
        The most probable macrostate for a system with \(q = 100\), \(N_A = 50\) and \(N_B = 
        50\) is \(50\). 
        The probability for this state is \(0.0562078\)

        

    \item

        Multiplicity of the Einstein model can be given by

        \begin{equation}
            \label{eq:2}
            \Omega(N,q) = \binom{q + N -1}{q} = \frac{(q + N - 1)!}{q!(N - 1)!}\\ 
        \end{equation}

        By using Stirlings approximation and the assumption that \(\frac{N}{q}<<1\)
        and \(N>>1\), the multiplicity can be simplified. 

        First of all, as we made the assumption that \(N>>1\), we can also assume that
        \(N -1 \approx N\). Then we are left with the following expression.

         \begin{equation}
            \label{eq:2}
            \Omega(N,q) \approx \frac{(q + N)!}{q!N!}\\ 
        \end{equation}   

        From this the expression can be written using \(ln\). 

        \begin{align*}
             ln\Omega(N,q) &\approx ln\frac{(q + N)!}{q!N!} = ln(q-N)! - ln(q!N!)\\
             & = ln(q + N)! - ln q! - ln N!
        \end{align*}

         With this expression, we can now use Stirlings approximation given by 
        \(ln x! = xlnx - x\). Which gives

        \begin{align*}
            ln\Omega &\approx (q+N)ln(q+N) - (q+N) - (q lnq - q) - (N lnN -N)\\
            &= (q + N)lnq(1 + \frac{N}{q}) - qlnq - NlnN \\ \\
            \frac{N}{q} << 1  &\rightarrow  lnq(1 + \frac{N}{q}) = lnq + \frac{N}{q} \\ \\
            ln\Omega &\approx N(ln \frac{q}{N} + 1)
        \end{align*}

    \item

        Since the multiplicity tends to be very large numbers, it is useful to work with
        the natural logarithm of the multiplicity instead of the multiplicity itself. 
        By multiplying the natural logarithm of the multiplicity with Boltzmann's constant,
        we get entropy, \(S = kln\Omega\). 

        From the previous task we have that the natural logarithm for an Einstein crystal 
        can be given by \(ln\Omega \approx N(ln \frac{q}{N} + 1) \). An expression for the
        entropy of an Einstein crystal is then

        \begin{equation*}
           S =  kN(ln \frac{q}{N} + 1)
        \end{equation*}

    \item

        The temperature of a system is given by \(T = \left ( \frac{\delta S}{\delta 
        U}\right ) ^{-
        1} \), where \(U = q\epsilon\). To find the temperature for an Einstein crystal, we 
        can use that
        \begin{equation*}
            \frac{\delta S}{\delta U} = \frac{\delta S}{\delta q} \frac{\delta q}{\delta U}
        \end{equation*}

        By calculating this we get
        \begin{align*}
            \frac{1}{T} &= \frac{1}{\epsilon} \frac{\delta}{\delta q}[kN(ln \frac{q}{N} + 
            1)]\\
            & = kN\frac{1}{q\epsilon} = \frac{kN}{U}\\ \\
            &\downarrow \\ \\
            T &= \left ( \frac{kN}{U} \right )^{-1}
        \end{align*}

        To say something about this, instead of looking at the temperature, we can look
        at U. Because we now have \(U = kNT\). The equipartition theorem says that the 
        the total energy should be \(\frac{1}{2}kT\) times the number of degrees of freedom. 
        As the Einstain crystal has two degrees of freedom for every oscillator, we see
        that our expression for temperature, and therefore for \(U\) fits this theorem. 

\section{Part 2: The spin system}


        We will now look at a paramagnetic system with binary spins. Each particle can
        be in two possible states, \(S = +1\) or \(S = -1\). The energy of a single particle
        is \(E = -S\mu B\). We will look at a system with N spins that do not interact. 

    \item

        The number of microstates for at N-spin system is easily calculated. As each 
        particle can have two different values and there are N particles, there will be 
        \(N^2\) microstates. 

    \item

        From now \(S_+\) will be used for the number of spins with value \(+1\) and \(S_-\) 
        as the number of spins with value \(-1\). We also introduce the net spin, \(2s = 
        S_+ - S_-\).

        An expression for the total energy, \(E\), as the function of the net spin
        is shown below. 

        \begin{equation*}
            E = \sum\limits_i E_i = \sum\limits_i -S_i \mu B = -2s \mu B
        \end{equation*}

    \item

        Assuming that all microstates are equally likely, we can generate M = 1000
        microstates for a N = 50 system randomly and plot the energies of the system.
        We can also plot a histogram for the energies for better visualization. 
        The program I used for doing this is the one below. The last part of the 
        program is for task q). 

        
        \lstinputlisting[language=Python, title = Finding and plotting energies 
        for a paramagnetic system with binary spins]{spin_system.py}  

        And the output relevant for this task is: 

         \begin{figure}[H]
            \centering % center the image horizontally
            \includegraphics[width=0.7\textwidth]{energy.png}
            \caption{Energy of the system plottet against the trial number} 
            \end{figure} 

            \begin{figure}[H]
            \centering % center the image horizontally
            \includegraphics[width=0.7\textwidth]{hist.png}
            \caption{Histogram of the energies of the system} 
            \end{figure}  

    \item

        The system has one macrostate for each possible \(S_+\) from \(0 \rightarrow N\),
        and the multiplicity will then be given by
        \begin{equation*}
            \Omega(N, S_+) = \binom{N}{S_+} = \frac{N!}{S_+!(N-S_+)}
        \end{equation*}

        When we consider that \(S_- = N - S_+\) we get the following expression

        \begin{equation*}
            \Omega(N, S_+) = \frac{N!}{S_+!S_-!}
        \end{equation*}

    \item

        The multiplicity can also be written as a function of the net spin, \(s\)
        \begin{equation*}
            \Omega(N, s) = \frac{N!}{\left (\frac{N}{2} + s \right )! \left (\frac{N}{2} - s 
        \right )!} 
        \end{equation*}

        We can show this by showing that this equation, and the equation from the previous
        task is the same. 
        I will start with looking at only \(\left (\frac{N}{2} + s \right )!\). 
        We know that \(s = \frac{S_+ - S_-}{2}\) and can therefore put this in,
        instead of s which gives

        \begin{equation*}
            \frac{N}{2} + \left (\frac{S_+ - S_-}{2} \right ) 
        \end{equation*}

        By recognizing that \(N - S_-\) is the same as \(S_+\) we get

        \begin{equation*}
            \frac{N}{2} + \left (\frac{S_+ - S_-}{2} \right ) = \frac{2S_+}{2} = S_+
        \end{equation*}

        We can then do the same calculation for \( \left (\frac{N}{2} - s 
        \right )! \)  and show that

        \begin{equation*}
            \left (\frac{N}{2} - s \right )! = \frac{2S_-}{2} = S_-
        \end{equation*} 

    \item

        CALCULATE STUPID STIRLING HERE!

    \item 
        
        By using the same program as in task m), we can compare the analytical
        result with the histogram generated. 

         \begin{figure}[H]
            \centering % center the image horizontally
            \includegraphics[width=0.7\textwidth]{analytical_approx.png}
            \caption{Histogram of the energies of the system compared with the analytical solution.} 
            \end{figure} 


    \item 

        The entropy is given by

        \begin{equation*}
            S(N,S_+) = kln\Omega(N, S_+) = k ln\left ( \frac{N!}{S_+! S_-!} \right )
        \end{equation*}

    \item

        The temperature is given by \(T = \left (\frac{\delta S}{\delta U} \right)^{-1} \). 
        From 
        task p) we remember that \(ln\Omega(N,s) = \Omega(N,0) - \frac{
        2s^2}{N} \). We also have that \(U = -2s \mu B\). Then we have

        \begin{align*}
            \frac{\delta S}{\delta U} &= \frac{\delta S}{\delta s} \frac{\delta s}{\delta 
            U}\\ \\
            &= - \frac{1}{2 \mu B} \frac{\delta}{\delta s} \left [k \left ( ln\Omega(N,0)
            - \frac{2s^2}{N} \right ) \right ] \\\\
            &= -\frac{1}{2 \mu B} \frac{4ks}{N} \\\\
            T &= - \frac{\mu BN}{2ks}
        \end{align*}

        If we want temperature expressed by \(S_+\) instead of \(s\) we know that 
        \(s = S_+ - \frac{N}{2}\). Which gives

        \begin{equation*}
            T = \frac{\mu BN}{2k\left (S_+ - \frac{N}{2}\right )}
        \end{equation*}

    \end{enumerate}


\end{document}                                             
                          
